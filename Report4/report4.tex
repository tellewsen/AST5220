% %% file: template.tex = LaTeX template for article-like report 
%% init: sometime 1993
%% last: Feb  8 2015  Rob Rutten  Deil
%% site: http://www.staff.science.uu.nl/~rutte101/rrweb/rjr-edu/manuals/student-report/

%% First read ``latex-bibtex-simple-manual.txt'' at
%% http://www.staff.science.uu.nl/~rutte101/Report_recipe.html

%% Start your report production by copying this file into your XXXX.tex.
%% Small changes to the header part will make it an A&A or ApJ manuscript.

%%%%%%%%%%%%%%%%%%%%%%%%%%%%%%%%%%%%%%%%%%%%%%%%%%%%%%%%%%%%%%%%%%%%%%%%%%%%
%\documentclass{aa}   %% Astronomy & Astrophysics style class
\documentclass[a4paper]{report}
\usepackage[inner=2cm,outer=2cm]{geometry}
%\geometry{a4paper}
\usepackage{graphicx,url,twoopt}
\usepackage{subcaption}
\usepackage{enumitem}
\usepackage{amsmath}
\usepackage[varg]{txfonts}           %% A&A font choice
%\usepackage{hyperref}                %% for pdflatex
%%\usepackage[breaklinks]{hyperref}  %% for latex+dvips
%%\usepackage{breakurl}              %% for latex+dvips
%\usepackage{pdfcomment}              %% for popup acronym meanings
%\usepackage{acronym}                 %% for popup acronym meanings
\usepackage{calrsfs}
\DeclareMathAlphabet{\pazocal}{OMS}{zplm}{m}{n}

\usepackage{natbib}
% \hypersetup{
%   colorlinks=true,   %% links colored instead of frames
%   urlcolor=blue,     %% external hyperlinks
%   linkcolor=red,     %% internal latex links (eg Fig)
% }

 %\bibpunct{(}{)}{;}{a}{}{,}    %% natbib cite format used by A&A and ApJ
 
%\pagestyle{plain}   %% undo the fancy A&A pagestyle 


%%%%%%%%%%%%%%%%%%%%%%%%%%%%%%%%%%%%%%%%%%%%%%%%%%%%%%%%%%%%%%%%%%%%%%%%%%%%
\begin{document}  

%\twocolumn[]
%\onecolumn
%%%%%%%%%%%%%%%%%%%%%%%%%%%%%%%%%%%%%%%%%%%%%%%%%%%%%%%%%%%%%%%%%%%%%%%%%%%%
\section{Introduction}\label{sec:introduction}
%%%%%%%%%%%%%%%%%%%%%%%%%%%%%%%%%%%%%%%%%%%%%%%%%%%%%%%%%%%%%%%%%%%%%%%%%%%%
In this project I am following the algorithm presented in Callin (2005)[1] for simulating the cosmic microwave background.  
This is the final part of the project.

In the first part I set up the background cosmology of the universe, and made a function that could find the conformal time as a function of $x$. In the second part I computed the electron fraction, electron density, optical depth and visibility function for times around and during recombination. The third part use the two previous to compute the density perturbations, and velocities of dark matter and baryons. This also included the temperature multipoles $\Theta_l$.

This final part combines all of these quantities to compute the final CMB power spectrum.

As previously done I will continue building on the skeleton code provided.

%%%%%%%%%%%%%%%%%%%%%%%%%%%%%%%%%%%%%%%%%%%%%%%%%%%%%%%%%%%%%%%%%%%%%%%%%%%%
\section{Equations}\label{sec:Equations}
%%%%%%%%%%%%%%%%%%%%%%%%%%%%%%%%%%%%%%%%%%%%%%%%%%%%%%%%%%%%%%%%%%%%%%%%%%%%

Transfer function
\begin{equation}
 \Theta_l(k,x=0)=\int^0_\infty \tilde{S}(k,x)j_l[k(\eta_0-\eta)]dx,
\end{equation}
where $j_l$ is the spherical Bessel functions, and S is the source function
 
%%%%%%%%%%%%%%%%%%%%%%%%%%%%%%%%%%%%%%%%%%%%%%%%%%%%%%%%%%%%%%%%%%%%%%%%%%%%
\section{Implementation}\label{sec:Imp}
%%%%%%%%%%%%%%%%%%%%%%%%%%%%%%%%%%%%%%%%%%%%%%%%%%%%%%%%%%%%%%%%%%%%%%%%%%%%

%%%%%%%%%%%%%%%%%%%%%%%%%%%%%%%%%%%%%%%%%%%%%%%%%%%%%%%%%%%%%%%%%%%%%%%%%%%%
\section{Conclusions}\label{sec:Conc}
%%%%%%%%%%%%%%%%%%%%%%%%%%%%%%%%%%%%%%%%%%%%%%%%%%%%%%%%%%%%%%%%%%%%%%%%%%%%
Unfortunately time did not allow me to use a Metropolis algorithm to estimate the values of the various cosmological parameters. This is something I will have to do on my own time afterwards. This is unfortunate as this would have been the icing on the cake.


%%%%%%%%%%%%%%%%%%%%%%%%%%%%%%%%%%%%%%%%%%%%%%%%%%%%%%%%%%%%%%%%%%%%%%%%%%%%
\section{References}
%%%%%%%%%%%%%%%%%%%%%%%%%%%%%%%%%%%%%%%%%%%%%%%%%%%%%%%%%%%%%%%%%%%%%%%%%%%%
\begin{enumerate}[label= {[}\arabic*{]} ]
 \item P. Callin, astro-ph/0606683
\end{enumerate}

\onecolumn 
%%%%%%%%%%%%%%%%%%%%%%%%%%%%%%%%%%%%%%%%%%%%%%%%%%%%%%%%%%%%%%%%%%%%%%%%%%%%
\section{Source code}\label{sec:files}
%%%%%%%%%%%%%%%%%%%%%%%%%%%%%%%%%%%%%%%%%%%%%%%%%%%%%%%%%%%%%%%%%%%%%%%%%%%%
The source code for the function made for computing the high resolution source function in evolution\_mod.f90 is included as well as the file cl\_mod.f90 file used for computing the final power spectrum is included for inspection. This file depends on all files previously used in the three earlier parts of the project.



%%%%%%%%%%%%%%%%%%%%%%%%%%%%%%%%%%%%%%%%%%%%%%%%%%%%%%%%%%%%%%%%%%%%%%%%%%%%
%\begin{acknowledgements}
%\end{acknowledgements}

\end{document}